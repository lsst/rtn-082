\documentclass[OPS,authoryear,toc]{lsstdoc}
\input{meta}

% Package imports go here.

% Local commands go here.

%If you want glossaries
%\input{aglossary.tex}
%\makeglossaries

\title{Pixel zone system security plan}

% This can write metadata into the PDF.
% Update keywords and author information as necessary.
\hypersetup{
    pdftitle={Pixel zone system security plan},
    pdfauthor={William O'Mullane},
    pdfkeywords={}
}

% Optional subtitle
% \setDocSubtitle{A subtitle}

\author{%
William O'Mullane
}

\setDocRef{RTN-082}
\setDocUpstreamLocation{\url{https://github.com/lsst/rtn-082}}

\date{\vcsDate}

% Optional: name of the document's curator
% \setDocCurator{The Curator of this Document}

\setDocAbstract{%
This document provides the mapping to NIST800-171 for the  Rubin pixel zone.
}

% Change history defined here.
% Order: oldest first.
% Fields: VERSION, DATE, DESCRIPTION, OWNER NAME.
% See LPM-51 for version number policy.
\setDocChangeRecord{%
  \addtohist{1}{YYYY-MM-DD}{Unreleased.}{William O'Mullane}
}


\begin{document}

% Create the title page.
\maketitle
% Frequently for a technote we do not want a title page  uncomment this to remove the title page and changelog.
% use \mkshorttitle to remove the extra pages

% ADD CONTENT HERE
% You can also use the \input command to include several content files.

\appendix
% Include all the relevant bib files.
% https://lsst-texmf.lsst.io/lsstdoc.html#bibliographies
\section{References} \label{sec:bib}
\renewcommand{\refname}{} % Suppress default Bibliography section
\bibliography{local,lsst,lsst-dm,refs_ads,refs,books}

% Make sure lsst-texmf/bin/generateAcronyms.py is in your path
\section{Acronyms} \label{sec:acronyms}
\addtocounter{table}{-1}
\begin{longtable}{p{0.145\textwidth}p{0.8\textwidth}}\hline
\textbf{Acronym} & \textbf{Description}  \\\hline

AAA & Authentication, Authorization and Accounting \\\hline
AC & Alternating Current \\\hline
AES & Advanced Encryption Standard \\\hline
AT & Auxiliary Telescope \\\hline
AU & deprecated acronym for astronomical unit; use au instead \\\hline
AURA & Association of Universities for Research in Astronomy \\\hline
CA & Control (or Cost) Account \\\hline
CCB & Change Control Board \\\hline
CM & Configuration Management \\\hline
CP & catalog prices \\\hline
CUI & Controlled Unclassified Information \\\hline
DM & Data Management \\\hline
DMTN & DM Technical Note \\\hline
DWDM & Dense Wave Division Multiplex \\\hline
EOL & End of Life \\\hline
IPA & FreeIPA - Identity, Policy, Audit \\\hline
IR & infrared \\\hline
ISO & Information Security Officer \\\hline
IT & Information Technology \\\hline
ITTN & IT Technote \\\hline
LHN & long haul network \\\hline
MAC & Media Access Control \\\hline
NDAA & National Defense Authorization Act \\\hline
NIST & National Institute of Standards and Technology (USA) \\\hline
NOIRLab & NSF's National Optical-Infrared Astronomy Research Laboratory; \url{https://noirlab.edu} \\\hline
NSF & National Science Foundation \\\hline
OPS & Operations \\\hline
OS & Operating System \\\hline
PS & Project Scientist \\\hline
PZ & photo-z \\\hline
RA & Right Ascension \\\hline
RTN & Rubin Technical Note \\\hline
S3 & (Amazon) Simple Storage Service \\\hline
SC & Science Collaboration \\\hline
SI & Syst\`eme International (International System of units defined by ISO) \\\hline
SLAC & SLAC National Accelerator Laboratory \\\hline
SP & Story Point \\\hline
SQR & SQuARE document handle \\\hline
SSID & Service Set Identifier \\\hline
SSL & Secure Sockets Layer \\\hline
USB & Universal Serial Bus \\\hline
USDF & United States Data Facility \\\hline
UTC & Coordinated Universal Time \\\hline
VPN & virtual private network \\\hline
VRO & (not to be used)Vera C. Rubin Observatory \\\hline
\end{longtable}

% If you want glossary uncomment below -- comment out the two lines above
%\printglossaries





\end{document}
