\tiny \begin{longtable} {|p{0.5\textwidth}|p{0.05\textwidth}|p{0.05\textwidth}|p{0.4\textwidth} |} \caption{This table provides an overview of the \citeds{NIST.SP.800-171} and Rubin compliance with it. \label{tab:compliance}}\\ 
\hline 
\textbf{NIST 800-171r3}&\textbf{2024 Status}&\textbf{Intended Compliance}&\textbf{Note} \\ \hline
{3.1 ACCESS CONTROL}&&& \\ \hline
{03.01.01 Account Management

a. Define the types of system accounts allowed and prohibited.

b. Create, enable, modify, disable, and remove system accounts in accordance with policy, procedures, prerequisites, and criteria.

c. Specify: 

1. Authorized users of the system,

2. Group and role membership, and

3. Access authorizations (i.e., privileges) for each account.

d. Authorize access to the system based on:

1. A valid access authorization and 

2. Intended system usage.

e. Monitor the use of system accounts.

f. Disable system accounts when:

1. The accounts have expired, 

2. The accounts have been inactive for [Assignment: organization-defined time period],

3. The accounts are no longer associated with a user or individual,

4. The accounts are in violation of organizational policy, or

5. Significant risks associated with individuals are discovered. 

g. Notify account managers and designated personnel or roles within:

1. [Assignment: organization-defined time period] when accounts are no longer required.

2. [Assignment: organization-defined time period] when users are terminated or transferred.

3. [Assignment: organization-defined time period] when system usage or the need-to-know changes for an individual.

h. Require that users log out of the system after [Assignment: organization-defined time period] of expected inactivity or when [Assignment: organization-defined circumstances].}&{Y}&{Y}&{IPA groups are in place for summit  which restrict privileges of individual users.
Off boarding and account disabling  in place - considering active account with monthly reaffirmation instead. 
See https://ittn-010.lsst.io/} \\ \hline
{03.01.02 Access Enforcement
Enforce approved authorizations for logical access to CUI and system resources in
accordance with applicable access control policies.}&{Y}&{Y}&{IPA groups are in place on the summit restricting users abilities. Legacy systems use the active directory groups for this.} \\ \hline
{03.01.03 Information Flow Enforcement
Enforce approved authorizations for controlling the flow of CUI within the system and between connected systems.}&{Y}&{Y}&{DMTN-199 defines the control flow for pixel data. Its implementation enforces it.} \\ \hline
{03.01.04 Separation of Duties

a. Identify the duties of individuals requiring separation.

b. Define system access authorizations to support separation of duties.}&{V}&{Y}&{Principle of least privilege is applied. Some users have access to hosts that is unneeded.} \\ \hline
{03.01.05 Least Privilege

a. Allow only authorized system access for users (or processes acting on behalf of users) that is necessary to accomplish assigned organizational tasks.

b. Authorize access to [Assignment: organization-defined security functions] and [Assignment: organization-defined security-relevant information].

c. Review the privileges assigned to roles or classes of users [Assignment: organization-defined frequency] to validate the need for such privileges.

d. Reassign or remove privileges, as necessary.}&{V}&{Y}&{IPA groups are in place for summit  which restrict privileges of individual users.
Off boarding and account disabling  in place - considering active account with monthly reaffirmation instead. 
See https://ittn-010.lsst.io/} \\ \hline
{03.01.06 Least Privilege – Privileged Accounts

a. Restrict privileged accounts on the system to [Assignment: organization-defined personnel or roles].

b. Require that users (or roles) with privileged accounts use non-privileged accounts when accessing non-security functions or non-security information.  
  }&{V}&{W}&{These accounts were specifically target in the Gemini attack - we would rather not use this approach.} \\ \hline
{03.01.07 Least Privilege – Privileged Functions

a. Prevent non-privileged users from executing privileged functions.

b. Log the execution of privileged functions.}&{Y}&{Y}&{a. sudo must be used for privileged functions

b. We log sudo attempts .} \\ \hline
{03.01.08 Unsuccessful Logon Attempts

a. Enforce a limit of [Assignment: organization-defined number] consecutive invalid logon attempts by a user during a [Assignment: organization-defined time period].

b. Automatically [Selection (one or more): lock the account or node for an [Assignment: organization-defined time period]; lock the account or node until released by an administrator; delay next logon prompt; notify system administrator; take other action] when the maximum number of unsuccessful attempts is exceeded.
}&{Y}&{Y}&{Web Services such as love, foreman, IPA console, nublado, etc. may need rate limiting. 
We don't use passwords in ssh hosts, only ssh keys (which many consider more secure).  
We are not aware of a retry limit for ssh-key access; an appropriate extra level of security would be to not use the default port 22. However, we do limit attempts to 6 with a block of 600 minutes, which will effectively block failed SUDO logins.} \\ \hline
{03.01.09 System Use Notification. 

Display a system use notification message with privacy and security notices
consistent with applicable CUI rules before granting access to the system.}&{N}&{Y}&{Check login notices etc. A login banner can be displayed upon login } \\ \hline
{03.01.10 Device Lock

a. Prevent access to the system by [Selection (one or more): initiating a device lock after [Assignment: organization-defined time period] of inactivity; requiring the user to initiate a device lock before leaving the system unattended].

b. Retain the device lock until the user reestablishes access using established identification and authentication procedures.

c. Conceal, via the device lock, information previously visible on the display with a publicly viewable image.}&{Y}&{Y}&{This is our policy.} \\ \hline
{03.01.11 Session Termination. 

Terminate a user session automatically after [Assignment: organization-defined
conditions or trigger events requiring session disconnect].}&{Y}&{Y}&{ssh sessions are generally not limited on hosts but VPN will timeout daily; some network equipment has timeouts set;} \\ \hline
{03.01.12 Remote Access

a. Establish usage restrictions, configuration requirements, and connection requirements for each type of allowable remote system access.

b. Authorize each type of remote system access prior to establishing such connections.

c. Route remote access to the system through authorized and managed access control points.

d. Authorize the remote execution of privileged commands and remote access to security-relevant information.}&{Y}&{Y}&{We currently check who and from where is connecting. IPA groups control access (and 2FA VPN). Bastion nodes are used to control ingress. UNIX groups are used at SLAC for access control.} \\ \hline
{03.01.13 Withdrawn}&{W}&{}&{Withdrawn in revision 3} \\ \hline
{03.01.14 Withdrawn}&{W}&{}&{Withdrawn in revision 3} \\ \hline
{03.01.15 Withdrawn}&{W}&{}&{Withdrawn in revision 3} \\ \hline
{03.01.16 Wireless Access

a. Establish usage restrictions, configuration requirements, and connection requirements for each type of wireless access to the system.

b. Authorize each type of wireless access to the system prior to establishing such connections.

c. Disable, when not intended for use, wireless networking capabilities prior to issuance and deployment.

d. Protect wireless access to the system using authentication and encryption.}&{Y}&{Y}&{All devices attaching in Chile need to be registered by Mac address. We  further consider still requiring 2FA VPN to access privileged systems from the WiFi.} \\ \hline
{03.01.17 Withdrawn}&{W}&{}&{Withdrawn in revision 3} \\ \hline
{03.01.18 Access Control for Mobile Devices

a. Establish usage restrictions, configuration requirements, and connection requirements for mobile devices.

b. Authorize the connection of mobile devices to the system.

c. Implement full-device or container-based encryption to protect the confidentiality of CUI on mobile devices.}&{Y}&{Y}&{Mobile devices must be registered on the summit - mobile devices do not contain pixel data.  In the case where an image may exist on say commissioning team laptop we will have disk encryption enabled. } \\ \hline
{03.01.19 Withdrawn}&{Y}&{Y}&{Withdrawn in revision 3} \\ \hline
{03.01.20 Use of External Systems

a. Prohibit the use of external systems unless the systems are specifically authorized.

b. Establish the following security requirements to be satisfied on external systems prior to allowing use of or access to those systems by authorized individuals: [Assignment: organization-defined security requirements].

c. Permit authorized individuals to use external systems to access the organizational system or to process, store, or transmit CUI only after:

1. Verifying that the security requirements on the external systems as specified in the organization’s system security plans have been satisfied and

2. Retaining approved system connection or processing agreements with the organizational entities hosting the external systems.

d. Restrict the use of organization-controlled portable storage devices by authorized individuals on external systems.}&{N}&{Y}&{We use mac address for laptops and personal mobile phones can not connect to the control network. 
We also have a separation with the LHN SSID and VLANs.
We do not allow external storage devices on the pixel zone.} \\ \hline
{03.01.21 Withdrawn}&{W}&{}&{Withdrawn in revision 3} \\ \hline
{03.01.22 Publicly Accessible Content

a. Train authorized individuals to ensure that publicly accessible information does not contain CUI.

b. Review the content on publicly accessible systems for CUI and remove such information, if discovered.}&{Y}&{Y}&{We do not intend to post images on publicly accessible systems. (DMTN-286). We intend to roll out training.} \\ \hline
{3.2 AWARENESS AND TRAINING}&&& \\ \hline
{03.02.01 Literacy Training and Awareness

a. Provide security literacy training to system users:

1. As part of initial training for new users and [Assignment: organization-defined frequency] thereafter,

2. When required by system changes or following [Assignment: organization-defined events], and

3. On recognizing and reporting indicators of insider threat, social engineering, and social mining.

b. Update security literacy training content [Assignment: organization-defined frequency] and following [Assignment: organization-defined events].
}&{Y}&{Y}&{A specific course for DMTN-199 is in prep. Each org has cyber security training already.} \\ \hline
{03.02.02 Role-Based Training

a. Provide role-based security training to organizational personnel:

1. Before authorizing access to the system or CUI, before performing assigned duties, and [Assignment: organization-defined frequency] thereafter

2. When required by system changes or following [Assignment: organization- defined events].

b. Update role-based training content [Assignment: organization-defined frequency] and following [Assignment: organization-defined events].}&{V}&{Y}&{OUO training at SLAC, DMTN-199 training for commissioners, Specific training for satellite catalog handlers.

We would like to do more here like capture flag exercises for developers or blue/red teams events.

Cyber training is annual.} \\ \hline
{03.02.03 Withdrawn}&{W}&{}&{Withdrawn in revision 3} \\ \hline
{3.3 AUDIT AND ACCOUNTABILITY}&&& \\ \hline
{03.03.01 Event Logging

a. Specify the following event types selected for logging within the system: [Assignment: organization-defined event types].

b. Review and update the event types selected for logging [Assignment: organization-defined frequency].}&{Y}&{Y}&{Observability contract.} \\ \hline
{03.03.02 Audit Record Content
a. Include the following content in audit records:

1. What type of event occurred

2. When the event occurred

3. Where the event occurred

4. Source of the event

5. Outcome of the event

6. Identity of the individuals, subjects, objects, or entities associated with the event

b. Provide additional information for audit records as needed.}&{Y}&{Y}& \\ \hline
{03.03.03 Audit Record Generation

a. Generate audit records for the selected event types and audit record content specified in 03.03.01 and 03.03.02.

b. Retain audit records for a time period consistent with the records retention policy.}&{Y}&{Y}&{Observability system} \\ \hline
{03.03.04 Response to Audit Logging Process Failures

a. Alert organizational personnel or roles within [Assignment: organization-defined time period] in the event of an audit logging process failure.

b. Take the following additional actions: [Assignment: organization-defined additional actions].}&{N}&{Y}& \\ \hline
{03.03.05 Audit Record Review, Analysis, and Reporting

a. Review and analyze system audit records [Assignment: organization-defined frequency] for indications and the potential impact of inappropriate or unusual activity.

b. Report findings to organizational personnel or roles.

c. Analyze and correlate audit records across different repositories to gain organization-wide situational awareness.}&{N}&{Y}&{Again shall look for third party contract for this} \\ \hline
{03.03.06 Audit Record Reduction and Report Generation

a. Implement an audit record reduction and report generation capability that supports audit record review, analysis, reporting requirements, and after-the-fact investigations of incidents.

b. Preserve the original content and time ordering of audit records.}&{Y}&{Y}&{Observability system} \\ \hline
{03.03.07 Time Stamps

a. Use internal system clocks to generate time stamps for audit records.

b. Record time stamps for audit records that meet [Assignment: organization-defined granularity of time measurement] and that use Coordinated Universal Time (UTC), have a fixed local time offset from UTC, or include the local time offset as part of the time stamp.}&{Y}&{Y}& \\ \hline
{03.03.08 Protection of Audit Information

a. Protect audit information and audit logging tools from unauthorized access, modification, and deletion.

b. Authorize access to management of audit logging functionality to only a subset of privileged users or roles.}&{Y}&{Y}&{Only specific admin users have access to audit logs} \\ \hline
{03.03.09 Withdrawn}&{W}&{}&{Withdrawn in revision 3} \\ \hline
{3.4 CONFIGURATION MANAGEMENT}&&& \\ \hline
{03.04.01 Baseline Configuration

a. Develop and maintain under configuration control, a current baseline configuration of the system.

b. Review and update the baseline configuration of the system [Assignment: organization-defined frequency] and when system components are installed or modified.}&{Y}&{Y}&{We use mainly infrastructure as code approaches so the software is well tracked. IT inventory all the hardware. } \\ \hline
{03.04.02 Configuration Settings

a. Establish, document, and implement the following configuration settings for the system that reflect the most restrictive mode consistent with operational requirements: [Assignment: organization-defined configuration settings].

b. Identify, document, and approve any deviations from established configuration settings.}&{Y}&{Y}&{Configuration settings are defined and documented in the lsst-it rancher, puppet  and phalanx repos. } \\ \hline
{03.04.03 Configuration Change Control

a. Define the types of changes to the system that are configuration-controlled.

b. Review proposed configuration-controlled changes to the system, and approve
or disapprove such changes with explicit consideration for security impacts.

c. Implement and document approved configuration-controlled changes to the
system.

d. Monitor and review activities associated with configuration-controlled changes
to the system.}&{Y}&{Y}&{We have an operations CCB (https://rtn-072.lsst.io/)  and code change process in place which also cover the infrastructure as code. } \\ \hline
{03.04.04 Impact Analyses

a. Analyze changes to the system to determine potential security impacts prior to change implementation.

b. Verify that the security requirements for the system continue to be satisfied after the system changes have been implemented.}&{Y}&{Y}&{Continuous integrations checks on puppet and phalanx check any changes prior to test deploy which is done prior to production. } \\ \hline
{03.04.05 Access Restrictions for Change

Define, document, approve, and enforce physical and logical access restrictions associated with changes to the system.}&{Y}&{Y}&{At infrastructure level this is is controlled by the Chile DevOps team.} \\ \hline
{03.04.06 Least Functionality

a. Configure the system to provide only mission-essential capabilities.

b. Prohibit or restrict use of the following functions, ports, protocols, connections, and services: [Assignment: organization-defined functions, ports, protocols, connections, and services].

c. Review the system [Assignment: organization-defined frequency] to identify unnecessary or nonsecure functions, ports, protocols, connections, and services.

d. Disable or remove functions, ports, protocols, connections, and services that are unnecessary or nonsecure.}&{Y}&{Y}&{Most application level functionality is controlled via phalanx. The OS level is puppet controlled.} \\ \hline
{03.04.07 Withdrawn}&{W}&{}&{Withdrawn in revision 3} \\ \hline
{03.04.08 Authorized Software – Allow by Exception

a. Identify software programs authorized to execute on the system.

b. Implement a deny-all, allow-by-exception policy for the execution of authorized software programs on the system.

c. Review and update the list of authorized software programs [Assignment: organization-defined frequency].}&{Y}&{Y}&{SUDO lists restrict access so users can not install applications on the summit nor in SLAC (outside a container). Mainly we containerize the applications and have users work within deployed containers. All containers are controlled/deployed via phalanx configuration.} \\ \hline
{03.04.09 Withdrawn}&{W}&{}&{Withdrawn in revision 3} \\ \hline
{03.04.10 System Component Inventory

a. Develop and document an inventory of system components.

b. Review and update the system component inventory [Assignment: organization-
defined frequency].

c. Update the system component inventory as part of installations, removals, and
system updates.}&{Y}&{Y}&{phalanx.lsst.io} \\ \hline
{03.04.11 Information Location

a. Identify and document the location of CUI and the system components on which
the information is processed and stored.

b. Document changes to the system or system component location where CUI is
processed and stored.}&{Y}&{Y}&{DMTN-199- Embargo rack and pixel zones are our places for restricted items.} \\ \hline
{03.04.12 System and Component Configuration for High-Risk Areas

a. Issue systems or system components with the following configurations to
individuals traveling to high-risk locations: [Assignment: organization-defined
system configurations].

b. Apply the following security requirements to the systems or components when
the individuals return from travel: [Assignment: organization-defined security
requirements].}&{N}&{Y}&{Though people self select to remove vaults and carry clean personal devices we do not have a strict policy nor do we have a list of high risk areas. In general there is no data on peoples machines so it is account/password vulnerability we would need to cover.} \\ \hline
{3.5 IDENTIFICATION AND AUTHENTICATION}&&& \\ \hline
{03.05.01 User Identification and Authentication

a. Uniquely identify and authenticate system users, and associate that unique identification with processes acting on behalf of those users.

b. Re-authenticate users when [Assignment: organization-defined circumstances or situations requiring re-authentication].}&{Y}&{Y}&{Users are associated with their unique Unix accounts.

Re-authentication is once per 24 hours. } \\ \hline
{03.05.02 Device Identification and Authentication

Uniquely identify and authenticate [Assignment: organization-defined devices or
types of devices] before establishing a system connection.}&{Y}&{Y}&{Users access via VPN with a 2FA device (DUO or 1password)} \\ \hline
{03.05.03 Multi-Factor Authentication

Implement multi-factor authentication for access to privileged and non-privileged accounts.}&{Y}&{Y}&{Summit uses 2FA - SLAC do no require this. } \\ \hline
{03.05.04 Replay-Resistant Authentication

Implement replay-resistant authentication mechanisms for access to privileged and non-privileged accounts.}&{Y}&{Y}&{Lockout after six failures.} \\ \hline
{03.05.05 Identifier Management

a. Receive authorization from organizational personnel or roles to assign an individual, group, role, service, or device identifier.

b. Select and assign an identifier that identifies an individual, group, role, service, or device.

c. Prevent the reuse of identifiers for [Assignment: organization-defined time period].

d. Manage individual identifiers by uniquely identifying each individual as [Assignment: organization-defined characteristic identifying individual status].}&{Y}&{Y}&{a. Jira tickets are used and management approval requested

b. Unique id is chosen

c. last 10 passwords can not be used

d. Single sign on across all systems uses same id.

See also https://ittn-045.lsst.io/} \\ \hline
{03.05.06 Withdrawn}&{W}&{}&{Withdrawn in revision 3} \\ \hline
{03.05.07 Password Management

a. Maintain a list of commonly-used, expected, or compromised passwords, and update the list [Assignment: organization-defined frequency] and when organizational passwords are suspected to have been compromised.

b. Verify that passwords are not found on the list of commonly used, expected, or compromised passwords when users create or update passwords.

c. Transmit passwords only over cryptographically protected channels.

d. Store passwords in a cryptographically protected form.

e. Select a new password upon first use after account recovery.

f. Enforce the following composition and complexity rules for passwords: [Assignment: organization-defined composition and complexity rules].}&{Y}&{Y}&{a. For the few system passwords we have  a generator is used such as 1password. 

b. We do use https://haveibeenpwned.com/Passwords

c. Passwords than must be shared are shared via 1password vaults. For users onetimesecret is used to pass an initial password which must then be replaced.

d. 1password is used for passwords

e. account recovery typically starts with a new password the user must then replace. 

f. complex passwords are required.} \\ \hline
{03.05.08 Withdrawn}&{W}&{}&{Withdrawn in revision 3} \\ \hline
{03.05.09 Withdrawn}&{W}&{}&{Withdrawn in revision 3} \\ \hline
{03.05.10 Withdrawn}&{W}&{}&{Withdrawn in revision 3} \\ \hline
{03.05.11 Authentication Feedback

Obscure feedback of authentication information during the authentication process.}&{Y}&{Y}&{Passwords are not echoed on any system.} \\ \hline
{03.05.12 Authenticator Management

a. Verify the identity of the individual, group, role, service, or device receiving the authenticator as part of the initial authenticator distribution.

b. Establish initial authenticator content for any authenticators issued by the organization.

c. Establish and implement administrative procedures for initial authenticator distribution; for lost, compromised, or damaged authenticators; and for revoking authenticators.

d. Change default authenticators at first use.

e. Change or refresh authenticators [Assignment: organization-defined frequency] or when the following events occur: [Assignment: organization-defined events].

f. Protect authenticator content from unauthorized disclosure and modification.}&{Y}&{Y}&{This applies mainly to passwords for us. We pass passwords with onetimesecret and then ask the user to change it immediately. } \\ \hline
{3.6 INCIDENT RESPONSE}&&& \\ \hline
{03.06.01 Incident Handling

Implement an incident-handling capability that is consistent with the incident
response plan and includes preparation, detection and analysis, containment,
eradication, and recovery. }&{Y}&{Y}&{Incident handling/response is in place. 

AURA also have insurance for serious incursions.} \\ \hline
{03.06.02 Incident Monitoring, Reporting, and Response Assistance

a. Track and document system security incidents.

b. Report suspected incidents to the organizational incident response capability within [Assignment: organization-defined time period].

c. Report incident information to [Assignment: organization-defined authorities].

d. Provide an incident response support resource that offers advice and assistance to system users on handling and reporting incidents.}&{Y}&{Y}&{We track and report incidents.

AURA insurance can provide further support if needed.} \\ \hline
{03.06.03 Incident Response Testing

Test the effectiveness of the incident response capability [Assignment: organization-defined frequency].}&{Y}&{Y}&{This was done at least with the PEN testing - which we shall repeat.} \\ \hline
{03.06.04 Incident Response Training

a. Provide incident response training to system users consistent with assigned roles and responsibilities:

1. Within [Assignment: organization-defined time period] of assuming an incident response role or responsibility or acquiring system access,

2. When required by system changes, and

3. [Assignment: organization-defined frequency] thereafter.

b. Review and update incident response training content [Assignment: organization-defined frequency] and following [Assignment: organization-defined events].}&{Y}&{Y}&{Cyber training includes user level incident response i.e. who to report attempts to. } \\ \hline
{03.06.05 Incident Response Plan

a. Develop an incident response plan that:

1. Provides the organization with a roadmap for implementing its incident response capability,

2. Describes the structure and organization of the incident response capability,

3. Provides a high-level approach for how the incident response capability fits into the overall organization,

4. Defines reportable incidents,

5. Addresses the sharing of incident information, and

6. Designates responsibilities to organizational entities, personnel, or roles.

b. Distribute copies of the incident response plan to designated incident response personnel (identified by name and/or by role) and organizational elements.

c. Update the incident response plan to address system and organizational changes or problems encountered during plan implementation, execution, or testing.

d. Protect the incident response plan from unauthorized disclosure.}&{Y}&{Y}&{RTN-030 Section 3.} \\ \hline
{3.7 MAINTENANCE}&&& \\ \hline
{03.07.01 Withdrawn}&{W}&{}&{Withdrawn in revision 3} \\ \hline
{03.07.02 Withdrawn}&{W}&{}&{Withdrawn in revision 3} \\ \hline
{03.07.03 Withdrawn}&{W}&{}&{Withdrawn in revision 3} \\ \hline
{03.07.04 Maintenance Tools

a. Approve, control, and monitor the use of system maintenance tools.

b. Check media with diagnostic and test programs for malicious code before it is used in the system.

c. Prevent the removal of system maintenance equipment containing CUI by verifying that there is no CUI on the equipment, sanitizing or destroying the equipment, or retaining the equipment within the facility.}&{Y}&{Y}&{a. Maintenance tools go through the requisition process - hence at least 2 managers approve. 

b. We run scans on downloaded media.

c. Maintenance equipment does not have CUI on it. } \\ \hline
{03.07.05 Nonlocal Maintenance

a. Approve and monitor nonlocal maintenance and diagnostic activities.

b. Implement multi-factor authentication and replay resistance in the establishment of nonlocal maintenance and diagnostic sessions.

c. Terminate session and network connections when nonlocal maintenance is completed.}&{Y}&{Y}&{a. Activities are always Jira ticketed

b. 2FA is always needed to access pixel zone.

c. Policy is to log off when done.} \\ \hline
{03.07.06 Maintenance Personnel

a. Establish a process for maintenance personnel authorization.

b. Maintain a list of authorized maintenance organizations or personnel.

c. Verify that non-escorted personnel who perform maintenance on the system possess the required access authorizations.}&{Y}&{Y}&{In general our staff do the maintenance. 
On occasion when we have remote assistance credentials are granted for a limited time and work is carried out with our staff. } \\ \hline
{3.8 MEDIA PROTECTION}&&& \\ \hline
{03.08.01 Media Storage

Physically control and securely store system media that contain CUI.}&{Y}&{Y}&{Pixel Zone and Embargo Rack} \\ \hline
{03.08.02 Media Access

Restrict access to CUI on system media to authorized personnel or roles.}&{Y}&{Y}&{Pixel Zone and Embargo Rack} \\ \hline
{03.08.03 Media Sanitization

Sanitize system media that contain CUI prior to disposal, release out of organizational control, or release for reuse.}&{Y}&{Y}&{We format/clean all devices prior to disposal/reuse.} \\ \hline
{03.08.04 Media Marking

Mark system media that contain CUI to indicate distribution limitations, handling
caveats, and applicable CUI markings.}&{Y}&{Y}&{We do not use any removable media for embargo information.} \\ \hline
{03.08.05 Media Transport

a. Protect and control system media that contain CUI during transport outside of controlled areas.

b. Maintain accountability of system media that contain CUI during transport outside of controlled areas.

c. Document activities associated with the transport of system media that contain CUI.}&{Y}&{Y}&{We do not use any removable media for embargo information. All transfers are over secure links.} \\ \hline
{03.08.06 Withdrawn}&{W}&{}&{Withdrawn in revision 3} \\ \hline
{03.08.07 Media Use

a. Restrict or prohibit the use of [Assignment: organization-defined types of system media].

b. Prohibit the use of removable system media without an identifiable owner.}&{N}&{Y}&{Can be rolled out with puppet but there are some servers require USB to be enabled but are in the server room. We can disable USB disk mounts at OS level. The machines and filesystem are encrypted so even if someone rebooted a node from a device to allow mounting USB they still could not get any data.} \\ \hline
{03.08.08 Withdrawn}&{W}&{}&{Withdrawn in revision 3} \\ \hline
{03.08.09 System Backup – Cryptographic Protection

a. Protect the confidentiality of backup information.

b. Implement cryptographic mechanisms to prevent the unauthorized disclosure of CUI at backup storage locations.}&{Y}&{Y}&{Pixel data is in only three locations - two in Chile and SLAC. There are no backups during embargo.} \\ \hline
{3.9 PERSONNEL SECURITY}&&& \\ \hline
{03.09.01 Personnel Screening

a. Screen individuals prior to authorizing access to the system.

b. Rescreen individuals in accordance with [Assignment: organization-defined conditions requiring rescreening].}&{Y}&{Y}&{Only project team members will have access to early images - all are known individuals screened on hiring. This doesn't suggest background security screening and it was also explicitly not required by the agencies in section 2 of the requirements document. } \\ \hline
{03.09.02 Personnel Termination and Transfer

a. When individual employment is terminated:

1. Disable system access within [Assignment: organization-defined time period],

2. Terminate or revoke authenticators and credentials associated with the individual, and

3. Retrieve security-related system property.

b. When individuals are reassigned or transferred to other positions in the organization:

1. Review and confirm the ongoing operational need for current logical and physical access authorizations to the system and facility, and

2. Modify access authorization to correspond with any changes in operational need.}&{Y}&{Y}&{This is the off boarding policy. Note that many collaborators retain some level of access even when off boarded.} \\ \hline
{3.10 PHYSICAL PROTECTION}&&& \\ \hline
{03.10.01 Physical Access Authorizations 

a. Develop, approve, and maintain a list of individuals with authorized access to the facility where the system resides.

b. Issue authorization credentials for facility access.

c. Review the facility access list [Assignment: organization-defined frequency].

d. Remove individuals from the facility access list when access is no longer required.}&{Y}&{Y}&{This physical access includes locks on server cabinets and key card access in base. (Contracted for summit computer room) } \\ \hline
{03.10.02 Monitoring Physical Access

a. Monitor physical access to the facility where the system resides to detect and respond to physical security incidents.

b. Review physical access logs [Assignment: organization-defined frequency] and upon occurrence of [Assignment: organization-defined events or potential indications of events].}&{Y}&{Y}&{Security is in place on Cerro Pachon and at the entrance to the mountain - though not only for Rubin so not permanently at the observatory. } \\ \hline
{03.10.03 Withdrawn}&{W}&{}&{Withdrawn in revision 3} \\ \hline
{03.10.04 Withdrawn}&{W}&{}&{Withdrawn in revision 3} \\ \hline
{03.10.05 Withdrawn}&{W}&{}&{Withdrawn in revision 3} \\ \hline
{03.10.06 Alternate Work Site

a. Determine alternate work sites allowed for use by employees.

b. Employ the following security requirements at alternate work sites: [Assignment: organization-defined security requirements].}&{Y}&{Y}&{All work can be done remotely from any location via the 2FA VPN. Cyber training assumes remote work is common.} \\ \hline
{03.10.07 Physical Access Control

a. Enforce physical access authorizations at entry and exit points to the facility where the system resides by:

1. Verifying individual physical access authorizations before granting access to the facility and

2. Controlling ingress and egress with physical access control systems, devices, or guards.

b. Maintain physical access audit logs for entry or exit points.

c. Escort visitors, and control visitor activity.

d. Secure keys, combinations, and other physical access devices.

e. Control physical access to output devices to prevent unauthorized individuals from obtaining access to CUI.}&{Y}&{Y}&{a.   Computer centers are restricted with key cards to appropriate staff - contractors are considered like staff. 

b. NOIRLab can currently store 80 gigs of data for audit logs of physical access, which will last at least three years -  all the equipment being installed is HID and complies with section 889 of the John S. McCain National Defense Authorization Act (NDAA)

c. visitors are escorted where appropriate i.e. where we have secure hardware. 

d. Individuals have cards/keys they are not left in insecure locations.

e. we will not be printing images.} \\ \hline
{03.10.08 Access Control for Transmission

Control physical access to system distribution and transmission lines within organizational facilities.}&{Y}&{Y}&{DWDM, secure routers are in card controlled room (summit contract pending)} \\ \hline
{3.11 RISK ASSESSMENT}&&& \\ \hline
{03.11.01 Risk Assessment

a. Assess the risk (including supply chain risk) of unauthorized disclosure resulting from the processing, storage, or transmission of CUI.

b. Update risk assessments [Assignment: organization-defined frequency].}&{Y}&{Y}&{This is part of our regular risk assessment process but we also look in depth at specific applications. Mostly we have concentrated the application exposure in phalanx which is carefully assessed and monitored.} \\ \hline
{03.11.02 Vulnerability Monitoring and Scanning

a. Monitor and scan the system for vulnerabilities [Assignment: organization- defined frequency] and when new vulnerabilities affecting the system are identified.

b. Remediate system vulnerabilities within [Assignment: organization-defined response times].

c. Update system vulnerabilities to be scanned [Assignment: organization-defined frequency] and when new vulnerabilities are identified and reported.}&{Y}&{Y}&{a. We monitor constantly also conduct third party contract PEN testing

b. We patch for vulnerabilities within 24 hours. 

c. third part applications are used for scanning} \\ \hline
{03.11.03 Withdrawn}&{W}&& \\ \hline
{03.11.04 Risk Response

Respond to findings from security assessments, monitoring, and audits.}&{Y}&{Y}&{We respond immediately to any security issue. It receives top priority.} \\ \hline
{3.12 SECURITY ASSESSMENT}&&& \\ \hline
{03.12.01 Security Assessment

Assess the security requirements for the system and its environment of operation [Assignment: organization-defined frequency] to determine if the requirements have been satisfied.}&{Y}&{Y}&{Annual reviews} \\ \hline
{03.12.02 Plan of Action and Milestones

a. Develop a plan of action and milestones for the system:

1. To document the planned remediation actions to correct weaknesses or deficiencies noted during security assessments and

2. To reduce or eliminate known system vulnerabilities.

b. Update the existing plan of action and milestones based on the findings from:

1. Security assessments,

2. Audits or reviews, and

3. Continuous monitoring activities.}&{Y}&{Y}&{We use Jira ticketing for all work including security patches and improvements.} \\ \hline
{03.12.03 Continuous Monitoring

Develop and implement a system-level continuous monitoring strategy that includes ongoing monitoring and security assessments.}&{Y}&{Y}&{Rubin is a mature organization with regular review and monitoring of all activities including cyber.} \\ \hline
{03.12.04 Withdrawn}&{W}&{}&{Withdrawn in revision 3} \\ \hline
{03.12.05 Information Exchange

a. Approve and manage the exchange of CUI between the system and other systems using [Selection (one or more): interconnection security agreements; information exchange security agreements; memoranda of understanding or agreement; service-level agreements; user agreements; non-disclosure agreements; other types of agreements].

b. Document interface characteristics, security requirements, and responsibilities for each system as part of the exchange agreements.

c. Review and update the exchange agreements [Assignment: organization-defined frequency].}&{Y}&{Y}&{This is entirely governed by DMTN-199 and its change control process.} \\ \hline
{3.13 SYSTEM AND COMMUNICATIONS PROTECTION}&&& \\ \hline
{03.13.01 Boundary Protection

a. Monitor and control communications at external managed interfaces to the system and key internal managed interfaces within the system.

b. Implement subnetworks for publicly accessible system components that are physically or logically separated from internal networks.

c. Connect to external systems only through managed interfaces that consist of boundary protection devices arranged in accordance with an organizational security architecture.}&{Y}&{Y}&{a. We have border scanning devices. 

b. We use VLANs and multiple VPNs to segment the network.

c. Bastions are used where needed and 2FA VPN for all users to connect to pixel zone.} \\ \hline
{03.13.02 Withdrawn}&{W}&{}&{Withdrawn in revision 3} \\ \hline
{03.13.03 Withdrawn}&{W}&{}&{Withdrawn in revision 3} \\ \hline
{03.13.04 Information in Shared System Resources

Prevent unauthorized and unintended information transfer via shared system resources.}&{Y}&{Y}&{DMTN-286 and SITCOMTN-076 cover ground rules on this} \\ \hline
{03.13.05 Withdrawn}&{W}&& \\ \hline
{03.13.06 Network Communications – Deny by Default – Allow by Exception

Deny network communications traffic by default, and allow network communications traffic by exception.}&{Y}&{Y}&{Routing and whitelisting is quite explicit.} \\ \hline
{03.13.07 Withdrawn}&{ }&{}&{Withdrawn in revision 3} \\ \hline
{03.13.08 Transmission and Storage Confidentiality

Implement cryptographic mechanisms to prevent the unauthorized disclosure of CUI during transmission and while in storage.}&{Y}&{Y}&{IPSec and encryption at rest. 2FA VPN to access summit.} \\ \hline
{03.13.09 Network Disconnect

Terminate the network connection associated with a communications session at the end of the session or after [Assignment: organization-defined time period] of inactivity.}&{Y}&{Y}&{We terminate connections after 24 hours} \\ \hline
{03.13.10 Cryptographic Key Establishment and Management

Establish and manage cryptographic keys in the system in accordance with the following key management requirements: [Assignment: organization-defined requirements for key generation, distribution, storage, access, and destruction].}&{Y}&{Y}& \\ \hline
{03.13.11 Cryptographic Protection

Implement the following types of cryptography to protect the confidentiality of CUI: [Assignment: organization-defined types of cryptography].}&{Y}&{Y}&{Disk encryption OS level and AES-256 on the wire.} \\ \hline
{03.13.12 Collaborative Computing Devices and Applications

a. Prohibit the remote activation of collaborative computing devices and applications with the following exceptions: [Assignment: organization-defined exceptions where remote activation is to be allowed].

b. Provide an explicit indication of use to users physically present at the devices.}&{Y}&{Y}&{This is our policy. } \\ \hline
{03.13.13 Mobile Code

a. Define acceptable mobile code and mobile code technologies.

b. Authorize, monitor, and control the use of mobile code.}&{Y}&{Y}&{Currently we have no mobile code} \\ \hline
{03.13.14 Withdrawn}&{W}&{}&{Withdrawn in revision 3} \\ \hline
{03.13.15 Session Authenticity

Protect the authenticity of communications sessions.}&{Y}&{Y}&{VPN and SSL/HTTPS connections are always used.} \\ \hline
{03.13.16 Withdrawn}&{W}&{}&{Withdrawn in revision 3} \\ \hline
{3.14 SYSTEM AND INFORMATION INTEGRITY}&&& \\ \hline
{03.14.01 Flaw Remediation

a. Identify, report, and correct system flaws.

b. Install security-relevant software and firmware updates within [Assignment: organization-defined time period] of the release of the updates.}&{Y}&{Y}&{Critical vulnerabilities are dealt with within 24 hours.} \\ \hline
{03.14.02 Malicious Code Protection

a. Implement malicious code protection mechanisms at system entry and exit points to detect and eradicate malicious code.

b. Update malicious code protection mechanisms as new releases are available in accordance with configuration management policies and procedures.

c. Configure malicious code protection mechanisms to:

1. Perform scans of the system [Assignment: organization-defined frequency] and real-time scans of files from external sources at endpoints or system entry and exit points as the files are downloaded, opened, or executed; and

2. Block malicious code, quarantine malicious code, or take other mitigation actions in response to malicious code detection.}&{Y}&{Y}& \\ \hline
{03.14.03 Security Alerts, Advisories, and Directives

a. Receive system security alerts, advisories, and directives from external organizations on an ongoing basis.

b. Generate and disseminate internal system security alerts, advisories, and directives, as necessary.}&{Y}&{Y}&{Handled by the ISO} \\ \hline
{03.14.04 Withdrawn}&{W}&{}&{Withdrawn in revision 3} \\ \hline
{03.14.05 Withdrawn}&{W}&{}&{Withdrawn in revision 3} \\ \hline
{03.14.06 System Monitoring

a. Monitor the system to detect:

1. Attacks and indicators of potential attacks and

2. Unauthorized connections.

b. Identify unauthorized use of the system.

c. Monitor inbound and outbound communications traffic to detect unusual or unauthorized activities or conditions.}&{Y}&{Y}&{Observability system} \\ \hline
{03.14.07 Withdrawn}&{W}&{}&{Withdrawn in revision 3} \\ \hline
{03.14.08 Information Management and Retention

Manage and retain CUI within the system and CUI output from the system in accordance with applicable laws, Executive Orders, directives, regulations, policies, standards, guidelines, and operational requirements.}&{Y}&{Y}&{DMTN-199 is the only applicable source.} \\ \hline
{3.15. Planning}&&& \\ \hline
{03.15.01 Policy and Procedures

a. Develop, document, and disseminate to organizational personnel or roles the policies and procedures needed to satisfy the security requirements for the protection of CUI.

b. Review and update policies and procedures [Assignment: organization-defined frequency].}&{Y}&{Y}& \\ \hline
{03.15.02 System Security Plan

a. Develop a system security plan that:

1. Defines the constituent system components;

2. Identifies the information types processed, stored, and transmitted by the system;

3. Describes specific threats to the system that are of concern to the organization;

4. Describes the operational environment for the system and any dependencies on or connections to other systems or system components;

5. Provides an overview of the security requirements for the system;

6. Describes the safeguards in place or planned for meeting the security requirements;

7. Identifies individuals that fulfill system roles and responsibilities; and

8. Includes other relevant information necessary for the protection of CUI.

b. Review and update the system security plan [Assignment: organization-defined frequency].

c. Protect the system security plan from unauthorized disclosure.}&{Y}&{Y}&{a. RTN-082

b. review at least annually

c. this is considered a public document} \\ \hline
{03.15.03 Rules of Behavior

a. Establish rules that describe the responsibilities and expected behavior for system usage and protecting CUI.

b. Provide rules to individuals who require access to the system.

c. Receive a documented acknowledgement from individuals indicating that they have read, understand, and agree to abide by the rules of behavior before authorizing access to CUI and the system.

d. Review and update the rules of behavior [Assignment: organization-defined frequency].}&{V}&{Y}&{Need new AUP} \\ \hline
{3.16. System and Services Acquisition}&&& \\ \hline
{03.16.01 Security Engineering Principles

Apply the following systems security engineering principles to the development or modification of the system and system components: [Assignment: organization- defined systems security engineering principles].}&{Y}&{Y}&{See RTN-082 Section 2.15} \\ \hline
{03.16.02 Unsupported System Components

a. Replace system components when support for the components is no longer available from the developer, vendor, or manufacturer.

b. Provide options for risk mitigation or alternative sources for continued support for unsupported components that cannot be replaced.}&{Y}&{Y}&{We keep uptodate and licensed.} \\ \hline
{03.16.03 External System Services

a. Require the providers of external system services used for the processing, storage, or transmission of CUI to comply with the following security requirements: [Assignment: organization-defined security requirements].

b. Define and document user roles and responsibilities with regard to external system services, including shared responsibilities with external service providers.

c. Implement processes, methods, and techniques to monitor security requirement compliance by external service providers on an ongoing basis.}&{Y}&{Y}&{a. No external providers are used for sensitive information.} \\ \hline
{3.17. Supply Chain Risk Management}&&& \\ \hline
{03.17.01 Supply Chain Risk Management Plan

a. Develop a plan for managing supply chain risks associated with the research and development, design, manufacturing, acquisition, delivery, integration, operations, maintenance, and disposal of the system, system components, or system services.

b. Review and update the supply chain risk management plan [Assignment: organization-defined frequency].

c. Protect the supply chain risk management plan from unauthorized disclosure.}&{N}&{W}&{Not applicable for this project.} \\ \hline
{03.17.02 Acquisition Strategies, Tools, and Methods

Develop and implement acquisition strategies, contract tools, and procurement methods to identify, protect against, and mitigate supply chain risks.  }&{N}&{W}&{Not applicable for this project.} \\ \hline
{03.17.03 Supply Chain Requirements and Processes

a. Establish a process for identifying and addressing weaknesses or deficiencies in the supply chain elements and processes.

b. Enforce the following security requirements to protect against supply chain risks to the system, system components, or system services and to limit the harm or consequences from supply chain-related events: [Assignment: organization- defined security requirements].}&{N}&{W}&{Not applicable for this project.} \\ \hline
\textbf{Total NIST800-171 requirements}&\textbf{}&\textbf{98}& \\ \hline
\textbf{Total Rubin Intends to comply fully with }&\textbf{}&\textbf{94}& \\ \hline
\textbf{Total Rubin Complies with in 2024}&\textbf{}&\textbf{84}& \\ \hline
\textbf{Total Rubin waivers requested }&\textbf{}&\textbf{4}& \\ \hline
\textbf{Total Rubin variances in 2024}&\textbf{}&\textbf{5}& \\ \hline
\end{longtable} \normalsize
